%%%%%%%%%%%%%%%%%%%%%%%%%%%%%%%%%%%%%%%%%
% LaTeX Template
% Version 1.2 (25/3/16)
%
% This template has been downloaded from:
% http://www.LaTeXTemplates.com
%
% Original author:
% Xavier Danaux (xdanaux@gmail.com) with modifications by:
% Vel (vel@latextemplates.com)
%
% License:
% CC BY-NC-SA 3.0 (http://creativecommons.org/licenses/by-nc-sa/3.0/)
%
% Important note:
% This template requires the moderncv.cls and .sty files to be in the same directory as this .tex file. These files provide the resume style and themes 
% used for structuring the document.
%
%%%%%%%%%%%%%%%%%%%%%%%%%%%%%%%%%%%%%%%%%

%----------------------------------------------------------------------------------------
%	PACKAGES AND OTHER DOCUMENT CONFIGURATIONS
%----------------------------------------------------------------------------------------

\documentclass[10pt,a4paper,sans]{moderncv} % Font sizes: 10, 11, or 12; paper sizes: a4paper, letterpaper, a5paper, legalpaper, executivepaper or landscape; font families: sans or roman

\moderncvstyle{banking}  % CV theme - options include: 'casual' (default), 'classic', 'oldstyle' and 'banking'
\moderncvcolor{blue}     % CV color - options include: 'blue' (default), 'orange', 'green', 'red', 'purple', 'grey' and 'black'
\usepackage{filecontents}

\usepackage[utf8]{inputenc}
\usepackage[T1]{fontenc}
\usepackage{cite}
\usepackage{lipsum} % Used for inserting dummy 'Lorem ipsum' text into the template

\usepackage[scale=0.80]{geometry}  % Reduce document margins

%----------------------------------------------------------------------------------------
%	NAME AND CONTACT INFORMATION SECTION
%----------------------------------------------------------------------------------------

\firstname{Maïmouna} % Your first name
\familyname{BOCOUM} % Your last name

% All information in this block is optional, comment out any lines you don't need
\title{Curriculum Vitae}
\address{95 rue du docteur Roux}{94100 Saint-Maur-des-Fossés, France}
\mobile{(+33) 6 14 06 41 50}
%\phone{(000) 111 1112}
%\fax{(000) 111 1113}
\email{physics@mbocoum.fr}
%\homepage{staff.org.edu/~jsmith}{staff.org.edu/$\sim$jsmith} % The first argument is the url for the clickable link, the second argument is the url displayed in the template - this allows special characters to be displayed such as the tilde in this example
\extrainfo{Permis B}
%\extrainfo{Date of birth: 01-27-1989}
%\photo[70pt][0.4pt]{pictures/picture.png} % The first bracket is the picture height, the second is the thickness of the frame around the picture (0pt for no frame)
%\quote{"A witty and playful quotation" - John Smith}

%----------------------------------------------------------------------------------------

\begin{document}

%----------------------------------------------------------------------------------------
%	COVER LETTER
%----------------------------------------------------------------------------------------

%% To remove the cover letter, comment out this entire block
%
%\clearpage
%
%\recipient{HR Department}{Corporation\\123 Pleasant Lane\\12345 City, State} % Letter recipient
%\date{\today} % Letter date
%\opening{Dear Sir or Madam,} % Opening greeting
%\closing{Sincerely yours,} % Closing phrase
%\enclosure[Attached]{curriculum vit\ae{}} % List of enclosed documents
%
%\makelettertitle % Print letter title
%
%\lipsum[1-3] % Dummy text
%
%\makeletterclosing % Print letter signature
%
%\newpage

%----------------------------------------------------------------------------------------
%	CURRICULUM VITAE
%----------------------------------------------------------------------------------------

\makecvtitle % Print the CV title

%----------------------------------------------------------------------------------------
%	EDUCATION SECTION
%----------------------------------------------------------------------------------------

\section{Éducation}

\cventry{2012--2016}{Doctorat}{École Polytechnique}{Paris}{}{Physique du cycle optique}
\cventry{2012}{Master}{École Polytechnique- UPMC}{Paris}{}{Sciences de la Fusion}  % Arguments not required can be left empty
\cventry{2009--2012}{Diplôme d'ingénieur}{ENSTA Paristech}{Paris}{}{Spécialisée en Physique et Mathématiques}

\section{Thèse de doctorat}

\cvitem{Titre}{Génération d’harmoniques d'ordres élevés et accélération d'électrons sur miroirs plasma}
\cvitem{Directeur de Thèse}{Rodrigo Lopez-Martens}
\cvitem{Description}{Lorsqu'une impulsions laser de quelques cycles optiques et d'intensité élevée ($\sim 10^{18}\mathrm{W/cm^2}$) interagit avec un plasma de densité solide, les électrons mis en mouvement génèrent, à chaque cycle, une impulsion attoseconde ($1\mathrm{as} = 10^{-18}\mathrm{s}$) dans l'X-UV. Cette émission cohérente étant périodique, le spectre d'émission X-UV est modulé par les harmoniques d'ordres élevés de l'impulsion laser initiale. Simultanément à la génération d'harmoniques, les champs très élevés ($\sim \mathrm{TV/m}$) mis en jeu lors de l’interaction conduisent à l'accélération d'électrons.}

%----------------------------------------------------------------------------------------
%	WORK EXPERIENCE SECTION
%----------------------------------------------------------------------------------------

\section{Expérience en Recherche}

%------------------------------------------------

\cventry{2017}
{Post-Doctorat}
{\textsc{Institut Langevin-INSERM}}
{Paris}
{}
{Développement d'une sonde exploitant l'effet Acousto-Optique (AO) pour l'imagerie médicale dans le cadre du projet MALT Plan Cancer. Travail expérimental sur le filtrage de fréquences optiques par holographie photoréfractive et hole-burning. Travail théorique sur la reconstruction d'images AO selon le type d'ondes acoustiques utilisées.}

\cventry{2016-2017}
{Post-Doctorat}
{\textsc{Laboratoire d'Optique Appliquée}}
{Palaiseau}
{}
{Travail expérimental sur la génération d'harmoniques en continuité du travail de thèse - conception d'une enceinte d’interaction pour l'installation européenne ELI-ALPS (Steged, Hongrie)}

\cventry{2012-2016}{Thèse}
{%institution
\textsc{Laboratoire d'Optique Appliquée}}
{%location
Palaiseau}
{}
{% details of the work
Thèse expérimentale sur la génération d'harmoniques d'ordres élevés sur miroirs plasmas. Conception d'une enceinte de post-compression, métrologie du cycle optique,  expériences pompe/sonde femtosecondes, spectroscopie d'électrons énergétiques ($\sim MeV$), stabilisation de cible solide kHz, programmation d'interfaces contrôle/commande Labview, traitement des données et développement de modèles analytiques, familiarisation avec les schémas de simulations numériques.
}

%------------------------------------------------

\cventry{2012}
{Stage de Master}
{\textsc{Laboratoire d'Utilisation des Lasers Intenses}}
{Palaiseau}
{}
{Campagne expérimentale de diffraction X pompe/sonde pour l'étude de la transition de phase du fer $\alpha$ soumis à de hautes pressions radiatives.}

%------------------------------------------------



\cventry{2010}{Stage d'été}{\textsc{École Polytechnique de Montréal}}
{Québec}
{}
{Caractérisation expérimentale du transport de charge dans la mélanine. Travail théorique sur la croissance de couches de tétracène et études de leurs propriétés polycristallines.}

%----------------------------------------------------------------------------------------
%	AWARDS SECTION
%----------------------------------------------------------------------------------------

\section{Bourses et Prix}

\cvitem{2012}{Bourse de thèse de l'ENSTA Paristech}
\cvitem{2014}{Prix de la meilleure présentation étudiante à la conférence ICUIL ("International Conference on Ultra Intense Lasers"), Goa-Inde}

%----------------------------------------------------------------------------------------
%	COMMUNICATION SKILLS SECTION
%----------------------------------------------------------------------------------------

\section{Enseignement}

\cvitem{2012 - 2018}{Chargée de TD de physique quantique en première année à l'ENSTA ParisTech}
\cvitem{2012 - 2015}{Chargée de TD d'optique non-linéaire en deuxième année à l'ENSTA ParisTech}
\cvitem{2008 – 2009}{Colles de Mathématiques en première année de classe préparatoire à l'école Michelet, Paris.}

%----------------------------------------------------------------------------------------
%	COMPUTER SKILLS SECTION
%----------------------------------------------------------------------------------------

\section{Informatique et électronique}

\cvitem{Basique}{\textsc{C/C++}, html, Linux, Arduino, Microprocesseurs mbed}
\cvitem{Intermédiaire}{LaTeX, Solidworks}
\cvitem{Avancé}{Matlab, Labview}


%----------------------------------------------------------------------------------------
%	LANGUAGES SECTION
%----------------------------------------------------------------------------------------

\section{Langues}

\cvitemwithcomment{Anglais}{Courant}{Expérience de 3 ans aux État-Unis}
\cvitemwithcomment{Espagnol}{Intermediaire}{Capable de tenir une conversation}
\cvitemwithcomment{Italien}{Intermediaire}{}
%----------------------------------------------------------------------------------------
%	INTERESTS SECTION
%----------------------------------------------------------------------------------------

\section{Publications}

\renewcommand{\listitemsymbol}{-~} % Changes the symbol used for lists

\begin{itemize}
\item J. Wünsche, G. Tarabella, S Bertolazz, \textbf{M.Bocoum} et al. "The correlation between gate dielectric, film growth, and charge transport in organic thin film transistors: the case of vacuum-sublimed tetracene thin films."  \textbf{Journal of Materials Chemistry C}  vol.1 no.5, pp967-976  (2013) 

\item W. Okell, T. Witting, D. Fabris, D. Austin, \textbf{M.Bocoum} et al. "Carrier-envelope phase stability of hollow fibers used for high-energy few-cycle pulse generation." \textbf{Optics letters} vol.38 no.20 pp3918-3021 (2013)

\item A. Denoeud, N. Osaki, A.Benuzzi-Mounaix, H. Uranishi, Y. Kondo, R. Kodamac, E. Brambrink, A. Ravasio, \textbf{M. Bocoum} et al. "Dynamic X-ray diffraction observation of shocked solid iron up to 170 GPa” \textbf{PNAS} vol.113 no.28 pp7745-7749 (2016)

\item \textbf{M. Bocoum} et al. "Spatial-domain interferometer for measuring plasma mirror expansion” \textbf{Optics letters} vol.40 no.13 pp3009-3012 (2015)

\item B. Beaurepaire, A. Vernier, \textbf{M.Bocoum} et al. "Effect of the laser wave front in a laser-plasma accelerator.” \textbf{Physical Review X} vol.5 no.3 pp.031012 (2015)

\item \textbf{M. Bocoum} et al. "Anticorrelated emission of high-order harmonics and fast electron beams for relativistic plasma mirrors”  \textbf{Physical Review Letters} vol.116 no.18” pp.185001  (2016)

\item  D. Guénot, D. Gustas, A. Vernier, B. Beaurepaire, F. Böhle, \textbf{M. Bocoum} and al. "Relativistic electron beams driven by kHz” \textbf{Nature Photonics} 11 pp293-296 (2017)

\item \textbf{M.Bocoum} (30ème auteur sur 42), “The eli-alps facility : the next generation of attosecond sources,” Journal of Physics B : Atomic, Molecular and Optical Physics vol.50 no.13 pp132002 (2017)

\item  \textbf{M. Bocoum} et al. “Two-color interpolation of absorption response for quantitative acousto-optic imaging,” \textbf{Optics letters} (en production) (2017)

\end{itemize}


%----------------------------------------------------------------------------------------
\end{document}